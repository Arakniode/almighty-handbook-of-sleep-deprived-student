\documentclass{Article}

\title{Énergie B \\ François Maréchal}
\author{Benjamin Bovey}


\begin{document}
\maketitle


\section{19 février 2019}
Nous avons vu le trilemne entre durabilité économique, sécuritaire (approvisionnement) et environnementale.

\section{26 février 2019}
Le charbon est relativement peu rare, on en a encore pour 135 ans avec les sources connues. La quantité n'est donc pas le problème, mais plutôt le fait que cela soit solide (plus difficile à stocker), et surtout qu'il pollue énormément (pollution dangereuse en Chine). Le charbon a également un impact énorme sur l'occupation des terres: la mine de charbon de Elsdorf fait 44km$^2$, 500m de profondeur. On y a creusé 40 millions de tonnes de matière par an depuis 20 ans. On a prévu d'en faire un lac artificiel qui se remplirait par la pluie, ce qui prendrait une centaine d'années selon les estimations. \\

 Le pétrole vient de restes organiques de petits animaux marins, qui furent enfermés sous une couche de roche imperméable. L'huile se trouve surtout entre 2000 et 3000 m de profondeur, le gaz plus profond. Le pétrole a une densité plus faible que celle de l'eau (il remonte à la surface). Le pétrole est une molécule complexe, que l'on doit traiter et séparer dans les raffineries de pétroles selon les différentes utilisations. Pour les différents types de pétroles raffinés, on a environ un rapport de 3 entre la quantité de carbone émise (plus haute) et la quantité de pétrole brulées. Entre 15 et 30\% de CO$_2$ sont émis avant la combustion. Il y a 20 à 30 millions de puits abandonnés autour du monde, qui fuyent, 1kg CH$_4$ = 25 kg CO$_2$, si brulé = 2.75 kg CO$_2$. On ne sait pas où ils sont, et dans quel état. \\

 Les gas de schiste sont extraits en faisant des puits souterrains horizontaux à travers les poches rocheuses, qui sont brisées par "fracking" (kaercher mon gars), ce qui consomme énormément d'eau qui finit chargée de minéraux (parfois radioactifs), qu'on va laisser évaporer à l'air libre. Un puit fonctionne pendant environ 7 ans, si tout va bien ==> il faut ensuite bouger tout l'équipement (en laissant le lac derrière). Si les puits sont mal faits, et que lors de la rencontre d'une nappe phréatique le tuyau n'est pas bien isolé, on peut contaminer les nappes avec du gaz (==> eau du robinet enflammable dans les environs). \\

 CO$_2$ émit avant la combustion: \\ \\
\begin{tabular}{ r | l }
	Charbon & moins de 3\% \\
	Diesel  & 15\% \\
	Gaz naturel & 19\% \\
	Gaz de schiste & 40\% \\
\end{tabular} \\

L'électricité est principalement produit par charbon (40\%). On chauffe de l'eau à très haute pression (120 bar), qui va activer des turbines qui transforment l'énergie mécanique en électricité (cycle de Rankine). L'eau est chauffée soit par combustion de carburant, soit par fission nucléaire. Une usine de 1000 MW consomme 45 tonnes d'uranium par ans, la même consomme 2'630'000 tonnes de charbon par an. L'uranium miné doit être préparé pour le réacteur par séparation chimique U3O8 -> UF6 (hexafluoride d'uranium).  On a un rendement d'environ 33\% sur la quantité de chaleur libérée par la fission. Une centrale nucléaire coûte environ 10c le kWh (100 euro / MWh), alors qu'en europe, le pris actuel est de 5c environ. Le démentèlement coûte environ aussi cher que la construction. Certains déchets ont une durée de vie d'environ 1 million d'années. \\

 Brûler du charbon est polluant, une bonne partie du système de combustion est dédiée à la gestion des poussières et souffre. Les turbines à gaz fonctionnent à peu près ainsi: on chauffe le gaz, qui se dilate, et fournit de l'énergie mécanique transformée en énergie électrique (rendement environ 40\%), on peut ensuite récupérer la chaleur résiduelle pour produire un peu plus d'électricité (rendement final environ 60\%). On peut imaginer des "virtual power plants", où chaque foyer peut produire sa propre énergie avec meilleur rendement que les grosses centrales (à rechercher). \\

Maintenant que nous connaissons le système énergéntique actuel, on peu s'interroger sur le futur.

\subsection{Comment la Suisse peut devenir énergiquement autonome?}

On a environ 47\% de consommation énergétique pour les habitations, 36\% pour les transports, et le reste pour ?. La ville représente environ 75\% de la population (truc avec déchets?). On estime que la population mondiale va plafonner à 11 milliards. Il faut comprendre où on perd de l'énergie (rénovation de l'habitat, minergie vs maison années 70). Cependant en raison du nombre de personnes + le nombre de metres carrés demandé par chaque individu, on va avoir beaucoup de couts de rénovation. Peut-on améliorer les chaudières? Selon Carnot, théoriquement, on devrait utiliser 1 énergie artificielle et 9 de l'environnement pour chauffer une maison à 25 degrés si l'extérieur est à 0 degrés (vérifier chiffres). L'énergie du soleil sur 1 année suffirait pour tous nos besoins en énergie pendant 6500 ans (si exploitée parfaitement). Elle varie cependant de l'heure, du lieu sur la planète, des conditions climatiques et météorologiques, de la saison... On peut designer les maisons de telle sorte à ce qu'elle fasse face au soleil en hiver, coté vitres ==> effet de serre, et moins en été. On peut concentrer l'énergie du soleil avec paraboles et autres ==> fermes solaires, cependant cela prend BEAUCOUP de surface (==> désert). Les panneaux solaires photovoltaïques sont de plus en plus efficaces (44\% rendement aujourd'hui en conditions idéales). Le photovoltaïque est aujourd'hui de loin le moyen le moins cher de poroduire de l'électricité dans le monde, avec des désavantages cités plus haut. L'émission de CO2 par panneau solaire: 40 à 50g CO2/kWhe, à comparer avec 1000g pour charbon. On ne peut pas stocker l'énergie en excès temporaire dans une batterie (il faudrait une batterie de la même taille que la maison). On peut utiliser un système à petits tuyaux où on échange du CO2 avec de la vapeur d'eau, refroidissement dans lacs naturels / eaux usées. Permet de chauffer / refroidir / récupérer de la chaleur résiduelle. L'électricité pour pompes à chaleur peut être produite par photovoltaïque. \\

 L'énergie éolienne a les mêmes inconvénients pratiques que le photovoltaïque (très variable). Il faut donc aussi penser à des systèmes d'échange/stockage.  \\

 La biomasse est en fait de l'énergie solaire stockée par les plantes. On peut la convertir en produisant de gaz (rendement jusqu'à 70\%). Le sous-produit de production de gaz est du CO2, mais en utilisant la réaction inverse à éléctrolyse on peut en faire du méthane (? VERIFIER).  \\

 La ville, avec les panneaux solaires sur les toits, peut devenir 100\% autonome en énergie. \\

 RESUME: on peut faire la même chose avec l'industrie et le transport, à condition qu'on ait une vision intégrée pou l'industrie utilise la biomasse, intègre la gestion des déchets, va être utilisée pour produire du gaz utilisé dans véhicules. Au total, la suisse peut devenir indépendante et neutre. On pourrait même produire plus de méthane que ce dont on a besoin (pourquoi besoin de méthane???). On peut même séquestrer le CO2 et transformer en CO3 qui est un solide inerte.

REMERCIEMENTS:
Soleil (énergie)
Mère Nature (stockage)
Carnot (efficacité)
industrie(technologies)
ringénierus(assemblr techno et utiliser)

recherche(formation, information)
(Autorités) système éducatif et développement infrastructure
(Finance) donner les moyens de réaliser la transition




\section{Le grand Tétémoin}
 Imagine if suddenly, there were no more banks in the world. What would happen to people? To businesses? Is this some kind of horror story? Wouldn't the world go to shit? \\

 The truth is, for half of the world population, banks do not exist. This is why we should, in fact, consider this seemingly horrible scenario. \\

 Many people borrow money to banks out of desperation, then loose everything. It has become accepted as a system, and has gotten a name: loanshark (banks that give and then take back). The idea that sparked it all came from wanting would protect people from these loansharks. He would lend money to people, without taking it back. Everyone loved it of course, and the idea spread from the village next to the university to the next, but his money was running out. In 1993, he was able to finally get the authorization to make a bank. In addition to being a bank for poor people, an innovative feature, for banks in Bangladesh (and around the world really), was that there was a will of having equity between men and women, and an equal repartition of the clients between men and women. Once that 50-50 was achieved, they realized that lending to women actually was much more effective socially, and decided to prioritize women. \\
 Banks usually say that you can't lend money to poor people, because poor people are "not worthy" of that money. The system went completely against all principles of regular banking. \\
The maximum loan is 1500 dollars in the US. For them, 1500 is a lot of money, especially for people living in big cities. They can use this money to start a small business. In the last 10 years, over a billion in loan, with a return of nearly exactly 100\%. The bank is lawyer-free: it is based on trust, there are no papers to sign like in usual banks. It becomes a "financial oxygen", because these loans allow people to get out of "financial dysfunction", commonly referred to as "poverty". This oxygen needs to exist because the current system that we built does not provide this oxygen to poor people. Poor people are bonzai people; they may not grow as big as the old forest trees, but if we give them the right place in a pot, they can grow and become a functional part of the ecosystem. Is the top-1\% system a good system for the people? No! The 1\% are not smarter or better; simply the 99\% weren't given a chance. \\

The 3 things that worry him:
\begin{itemize}
	\item wealth concentration
	\item environment
	\item technology
\end{itemize}

The 1st one we already talked about. The 2nd one is more worrying because of inaction; by 2050 we will have passed the point of no return. Why is there so much inaction? Because we don't care! The attention is taken by the chase for money. We are being irresponsible. An action has to be taken to undo the things we have been doing for 100s of years. \\

The 3rd one is technology. We always pride it: it promises us future! But it has become a threat. Because of AI. What implications does it have that are "bad"? Answer: they remplace human beings. Machines have always been made to remplace a part of the human being. Today we are making machines so general and/or diverse they could remplace the human being. Already, they are winning games, writing books, painting works of art. The question is: what happens to human beings? The usual answer is: don't worry. We have arrived at the peak of humanity. Machines should work, humans should enjoy. But who's bringing the food to the table? "Well, all you have to do is say, the government provides everyone money, so they can buy food." But I'm an human being. I pride myself of being able to do things for myself, but also for the world. I don't want machines to take care of that. Plus, what will the relationship be between the humans, and the machines 20x more intelligent? It cannot be equal. They will end up finding these human beings to be nuisances. Just like we consider bugs. \\
What does this mean? Technology can be good, technology can be bad. The way that AI is being developed today is wrong. We have to draw the red line. We have to only use it for the truly essential tasks.




















\end{document}