\documentclass{article}

\title{Énergie B \\ Notes}
\author{Benjamin Bovey}

% the next 3 lines are for removing section numbering. You may want to comment them out.
\makeatletter
\renewcommand{\@seccntformat}[1]{}
\makeatother

\usepackage[utf8]{inputenc}

\begin{document}
\maketitle


\section{19 février 2019}
Nous avons vu le trilemne entre durabilité économique, sécuritaire (approvisionnement) et environnementale.

\section{26 février 2019}
Le charbon est relativement peu rare, on en a encore pour 135 ans avec les sources connues. La quantité n'est donc pas le problème, mais plutôt le fait que cela soit solide (plus difficile à stocker), et surtout qu'il pollue énormément (pollution dangereuse en Chine). Le charbon a également un impact énorme sur l'occupation des terres: la mine de charbon de Elsdorf fait 44km$^2$, 500m de profondeur. On y a creusé 40 millions de tonnes de matière par an depuis 20 ans. On a prévu d'en faire un lac artificiel qui se remplirait par la pluie, ce qui prendrait une centaine d'années selon les estimations. \\

 Le pétrole vient de restes organiques de petits animaux marins, qui furent enfermés sous une couche de roche imperméable. L'huile se trouve surtout entre 2000 et 3000 m de profondeur, le gaz plus profond. Le pétrole a une densité plus faible que celle de l'eau (il remonte à la surface). Le pétrole est une molécule complexe, que l'on doit traiter et séparer dans les raffineries de pétroles selon les différentes utilisations. Pour les différents types de pétroles raffinés, on a environ un rapport de 3 entre la quantité de carbone émise (plus haute) et la quantité de pétrole brulées. Entre 15 et 30\% de CO$_2$ sont émis avant la combustion. Il y a 20 à 30 millions de puits abandonnés autour du monde, qui fuyent, 1kg CH$_4$ = 25 kg CO$_2$, si brulé = 2.75 kg CO$_2$. On ne sait pas où ils sont, et dans quel état. \\

 Les gas de schiste sont extraits en faisant des puits souterrains horizontaux à travers les poches rocheuses, qui sont brisées par "fracking" (kaercher mon gars), ce qui consomme énormément d'eau qui finit chargée de minéraux (parfois radioactifs), qu'on va laisser évaporer à l'air libre. Un puit fonctionne pendant environ 7 ans, si tout va bien ==> il faut ensuite bouger tout l'équipement (en laissant le lac derrière). Si les puits sont mal faits, et que lors de la rencontre d'une nappe phréatique le tuyau n'est pas bien isolé, on peut contaminer les nappes avec du gaz (==> eau du robinet enflammable dans les environs). \\

 CO$_2$ émit avant la combustion: \\ \\
\begin{tabular}{ r | l }
	Charbon & moins de 3\% \\
	Diesel  & 15\% \\
	Gaz naturel & 19\% \\
	Gaz de schiste & 40\% \\
\end{tabular} \\

L'électricité est principalement produit par charbon (40\%). On chauffe de l'eau à très haute pression (120 bar), qui va activer des turbines qui transforment l'énergie mécanique en électricité (cycle de Rankine). L'eau est chauffée soit par combustion de carburant, soit par fission nucléaire. Une usine de 1000 MW consomme 45 tonnes d'uranium par ans, la même consomme 2'630'000 tonnes de charbon par an. L'uranium miné doit être préparé pour le réacteur par séparation chimique U3O8 -> UF6 (hexafluoride d'uranium).  On a un rendement d'environ 33\% sur la quantité de chaleur libérée par la fission. Une centrale nucléaire coûte environ 10c le kWh (100 euro / MWh), alors qu'en europe, le pris actuel est de 5c environ. Le démentèlement coûte environ aussi cher que la construction. Certains déchets ont une durée de vie d'environ 1 million d'années. \\

 Brûler du charbon est polluant, une bonne partie du système de combustion est dédiée à la gestion des poussières et souffre. Les turbines à gaz fonctionnent à peu près ainsi: on chauffe le gaz, qui se dilate, et fournit de l'énergie mécanique transformée en énergie électrique (rendement environ 40\%), on peut ensuite récupérer la chaleur résiduelle pour produire un peu plus d'électricité (rendement final environ 60\%). On peut imaginer des "virtual power plants", où chaque foyer peut produire sa propre énergie avec meilleur rendement que les grosses centrales (à rechercher). \\

Maintenant que nous connaissons le système énergéntique actuel, on peu s'interroger sur le futur.

\subsection{Comment la Suisse peut-elle devenir énergiquement autonome?}

On a environ 47\% de consommation énergétique pour les habitations, 36\% pour les transports, et le reste pour ?. La ville représente environ 75\% de la population (truc avec déchets?). On estime que la population mondiale va plafonner à 11 milliards. Il faut comprendre où on perd de l'énergie (rénovation de l'habitat, minergie vs maison années 70). Cependant en raison du nombre de personnes + le nombre de metres carrés demandé par chaque individu, on va avoir beaucoup de couts de rénovation. Peut-on améliorer les chaudières? Selon Carnot, théoriquement, on devrait utiliser 1 énergie artificielle et 9 de l'environnement pour chauffer une maison à 25 degrés si l'extérieur est à 0 degrés (vérifier chiffres). L'énergie du soleil sur 1 année suffirait pour tous nos besoins en énergie pendant 6500 ans (si exploitée parfaitement). Elle varie cependant de l'heure, du lieu sur la planète, des conditions climatiques et météorologiques, de la saison... On peut designer les maisons de telle sorte à ce qu'elle fasse face au soleil en hiver, coté vitres ==> effet de serre, et moins en été. On peut concentrer l'énergie du soleil avec paraboles et autres ==> fermes solaires, cependant cela prend BEAUCOUP de surface (==> désert). Les panneaux solaires photovoltaïques sont de plus en plus efficaces (44\% rendement aujourd'hui en conditions idéales). Le photovoltaïque est aujourd'hui de loin le moyen le moins cher de poroduire de l'électricité dans le monde, avec des désavantages cités plus haut. L'émission de CO2 par panneau solaire: 40 à 50g CO2/kWhe, à comparer avec 1000g pour charbon. On ne peut pas stocker l'énergie en excès temporaire dans une batterie (il faudrait une batterie de la même taille que la maison). On peut utiliser un système à petits tuyaux où on échange du CO2 avec de la vapeur d'eau, refroidissement dans lacs naturels / eaux usées. Permet de chauffer / refroidir / récupérer de la chaleur résiduelle. L'électricité pour pompes à chaleur peut être produite par photovoltaïque. \\

 L'énergie éolienne a les mêmes inconvénients pratiques que le photovoltaïque (très variable). Il faut donc aussi penser à des systèmes d'échange/stockage.  \\

 La biomasse est en fait de l'énergie solaire stockée par les plantes. On peut la convertir en produisant de gaz (rendement jusqu'à 70\%). Le sous-produit de production de gaz est du CO2, mais en utilisant la réaction inverse à éléctrolyse on peut en faire du méthane (? VERIFIER).  \\

 La ville, avec les panneaux solaires sur les toits, peut devenir 100\% autonome en énergie. \\

 RESUME: on peut faire la même chose avec l'industrie et le transport, à condition qu'on ait une vision intégrée pou l'industrie utilise la biomasse, intègre la gestion des déchets, va être utilisée pour produire du gaz utilisé dans véhicules. Au total, la suisse peut devenir indépendante et neutre. On pourrait même produire plus de méthane que ce dont on a besoin (pourquoi besoin de méthane???). On peut même séquestrer le CO2 et transformer en CO3 qui est un solide inerte.

REMERCIEMENTS:
Soleil (énergie)
Mère Nature (stockage)
Carnot (efficacité)
industrie(technologies)
ringénierus(assemblr techno et utiliser)

recherche(formation, information)
(Autorités) système éducatif et développement infrastructure
(Finance) donner les moyens de réaliser la transition




\section{Le grand Tétémoin}
 Imagine if suddenly, there were no more banks in the world. What would happen to people? To businesses? Is this some kind of horror story? Wouldn't the world go to shit? \\

 The truth is, for half of the world population, banks do not exist. This is why we should, in fact, consider this seemingly horrible scenario. \\

 Many people borrow money to banks out of desperation, then loose everything. It has become accepted as a system, and has gotten a name: loanshark (banks that give and then take back). The idea that sparked it all came from wanting would protect people from these loansharks. He would lend money to people, without taking it back. Everyone loved it of course, and the idea spread from the village next to the university to the next, but his money was running out. In 1993, he was able to finally get the authorization to make a bank. In addition to being a bank for poor people, an innovative feature, for banks in Bangladesh (and around the world really), was that there was a will of having equity between men and women, and an equal repartition of the clients between men and women. Once that 50-50 was achieved, they realized that lending to women actually was much more effective socially, and decided to prioritize women. \\
 Banks usually say that you can't lend money to poor people, because poor people are "not worthy" of that money. The system went completely against all principles of regular banking. \\
The maximum loan is 1500 dollars in the US. For them, 1500 is a lot of money, especially for people living in big cities. They can use this money to start a small business. In the last 10 years, over a billion in loan, with a return of nearly exactly 100\%. The bank is lawyer-free: it is based on trust, there are no papers to sign like in usual banks. It becomes a "financial oxygen", because these loans allow people to get out of "financial dysfunction", commonly referred to as "poverty". This oxygen needs to exist because the current system that we built does not provide this oxygen to poor people. Poor people are bonzai people; they may not grow as big as the old forest trees, but if we give them the right place in a pot, they can grow and become a functional part of the ecosystem. Is the top-1\% system a good system for the people? No! The 1\% are not smarter or better; simply the 99\% weren't given a chance. \\

The 3 things that worry him:
\begin{itemize}
	\item wealth concentration
	\item environment
	\item technology
\end{itemize}

The 1st one we already talked about. The 2nd one is more worrying because of inaction; by 2050 we will have passed the point of no return. Why is there so much inaction? Because we don't care! The attention is taken by the chase for money. We are being irresponsible. An action has to be taken to undo the things we have been doing for 100s of years. \\

The 3rd one is technology. We always pride it: it promises us future! But it has become a threat. Because of AI. What implications does it have that are "bad"? Answer: they remplace human beings. Machines have always been made to remplace a part of the human being. Today we are making machines so general and/or diverse they could remplace the human being. Already, they are winning games, writing books, painting works of art. The question is: what happens to human beings? The usual answer is: don't worry. We have arrived at the peak of humanity. Machines should work, humans should enjoy. But who's bringing the food to the table? "Well, all you have to do is say, the government provides everyone money, so they can buy food." But I'm an human being. I pride myself of being able to do things for myself, but also for the world. I don't want machines to take care of that. Plus, what will the relationship be between the humans, and the machines 20x more intelligent? It cannot be equal. They will end up finding these human beings to be nuisances. Just like we consider bugs. \\
What does this mean? Technology can be good, technology can be bad. The way that AI is being developed today is wrong. We have to draw the red line. We have to only use it for the truly essential tasks.


\section{12 mars 2019 \quad Fusion nucléaire et véhicules électriques}
\subsection{L'énergie de fusion, ou comment créer une étoile sur terre.}
La recherche sur la fusion s'inscrit dans l'idée de la compatibilité avec le développement durable: ne pas mettre d'hypothèque sur les générations futures.
Plan:
\begin{enumerate}
	\item Energie du noyau (fission et fusion, $E = mc^2$)
	\item Fusion (inertielle, à confinement magnétique)
	\item Le tokamak (principe de fonctionnement)
	\item Le futur: ITER et DEMO
\end{enumerate}

\textbf{Fission}: La fission consiste à briser un gros atome (donc relativement instable et énergétique) en le bombardant d'électrons, afin qu'il devienne plus stable après en relâchant de l'énergie. \par
\textbf{Fusion}: La fusion consiste à mettre ensemble (fusionner) des petits noyaux d'atomes. Les produits finaux sont plus légers et relâchent de l'énergie ($E = mc^2$) pendant leur transformation (??? pourquoi) \par 
Une grande énergie cinétique est nécessaire pour fusionner deux noyaux, dû à la force de Coulomb (je crois) qui fait que les particules de même charge se repoussent (les protons dans ce cas). \par
À 100'000'000 degrés celsius, la matière est à l'état plasmique (le quatrième état de la matière). La plupart de l'univers connu est en fait composé de plasma. Il peut être utilisé dans des tonnes de domaines: lumière, moteurs de fusées, médecine, etc. Le potentiel du plasma dans le domaine de l'énergie est énorme:
\begin{enumerate}
	\item la densité d'énergie est immense. Là où les combustibles "classiques" (charbon, pétrole) peuvent émettre 20 à 40 MJ/kg, la fusion deutérium-tritium donne 350'000'000 (l'énergie totale donnée selon $E = mc^2$ est de 90'000'000'000).
	\item Utilise des ressources pratiquement inépuisables (l'eau).
	\item Pas de déchets radioactifs de longue durée, pas d'émission de gaz à effet de serre.
	\item Pas de risque de perte de contrôle des réactions (pas de réactions en chaîne), quantité minime de combustile dans le réacteur (environ 1g), et pas de lien avec la prolifération d'armement nucléaire.
	\item Énergie concentrée, donc pas dépendante de facteurs naturels.
\end{enumerate}

La question se pose cependant du confinement du plasma: les étoiles le conservent en leur centre par l'énorme force gravitationnelle due à leur masse. Sur Terre, c'est impossible. Cependant, on a pensé principalement à deux solutions: \par
\textbf{Confinement inertiel}:  Il consiste à garder le plasma uniquement pour de très courtes durées, et d'exploiter au mieux l'énergie pendant cette fenêtre très courte. On "tape" avec des lasers très puissants sur des pilules de matière. Il vont élever cette masse au-dessus de la température critique de 100'000'000 de degrés. Cela crée effectivement une miniscule étoile/bombe, durant quelques milliardièmes de secondes (environ 20). Pour l'instant, il semble difficile de créer un réacteur avec cette technologie, comme il faudrait répéter ce processus environ 10 fois par seconde, et chaque pilule coûte actuellement 40'000 dollars. Cependant, on pourrait très bien voir cette technologie utilisée dans l'armement, les départements de la défense autour du monde soutenant d'ailleurs ces recherches. \par
\textbf{Confinement magnétique}: Ici, l'idée est de garder le plasma sur place pendant longtems (on parle de durées de plusieurs secondes, ou plus????). Sur Terre, le champ magnétique nous protège des plasmas spatiaux (aux pôles, le champ magnétique est plus faible ce qui engendre parfois des aurores boréales). On peut donc utiliser ce même méchanisme afin de confiner du plasma. On le piège dans un tore avec un champ magnétique "s'enroulant" autour du tore (le plasma est un excellent conducteur). Ce type de système a été nommé "Tokamak", qui est un nom russe, et il existe actuellement une quarantaine de tokamaks autour du monde, en utilisation ou en construction. Le concept est né à la fin des années 50, et le développement a été rapide depuis. Actuellement, on atteint un rendement d'énergie de 1 pour 1 (donc pas encore compétitif). \par
 Le futur s'appelle ITER ou DEMO. ITER est la preuve de la faisabilité scientifique et technologique de la fusion. Il est actuellement en construction au sud de la France. Le projet ITER a été annoncé à la fin de la guerre froide par Reagan et l'autre russe là. Il est le résultat d'une collaboration globale, où chaque parti de l'accord fournit une partie de l'appareillage. Un des projets les ambitieux faits sur Terre, se plaçant en 3e place derrière l'ISS (1er) et Apollo (2e). \par
 Cependant, si ITER démontre la faisabilité de la fusion nucléaire, c'est DEMO qui faira le dernier pas vers la commercialisation. \par
 Les défis scientifiques sont:
\begin{itemize}
	\item Manteau pour la génération de tritium
	\item Robotique: remote handling
	\item Aimants supraconducteurs (qui sont sensés être à extrêmement basse température pour fonctionner, et se trouvent juste à côté d'une zone à température plus haute que celle du soleil!)
	\item Chauffage du plasma et conditions stationnaires
	\item Confinement du plasma et turbulence
	\item Contrôle du plasma et ses instabilités
	\item Bord du plasma et interactions plasma-parois
\end{itemize}
 
PS: il y a une raison au vent chaud à l'entrée des magasins en hiver: ça permet d'empêcher une trop grande fuite de chaleur (étonamment)!


\subsection{Véhicules électriques: tendances et défis}
Le véhicule électrique n'est pas quelque chose de nouveau. Déjà, ils séduisaient au début du 20e siècle car ils ne requiéraient pas de démarrage à la manivelle. General Motors on retiré dans les années 2000 tous leurs véhicules électriques du marché sous pression des marchés pétroliiers (à voir: "Who killed the electric car?"). Ce mode de transport on donc déjà connu deux morts, mais la génération apparue dans les années 2008 sera-t-elle la bonne? \par
À partir de 2010, de nouveaux modèles arrivent sur le marché: Mitsubishi, Renault, Ford, Mercedes, la première Tesla (Roadster), tous rejoignent la nouvelle course. Sur le marché, une cinquantaine de véhicules sont actuellement en vente autour du monde. Le nombre de bornes de rechargement a grandement augmenté, ce qui est un facteur important à la propagation des voitures elles-mêmes (les prises c'est le bordel cependant: par exemple les voitures Tesla peut aller chez tout le monde, mais personne ne peut aller chez Tesla). \par
 Mais qu'est-ce qui a changé depuis la dernière mort de la voiture électrique? 
\begin{itemize}
	\item Le contexte macro-économique (passage du Oil Peak, inquiétudes croissantes quant à la sécurité énergétique)
	\item Le progrès technologique (grandes améliorations dans le domaine des batteries)
	\item Le contexte de la prise de conscience autour du réchauffement climatique (conclusions de l'IPCC, problématique des NOX et particules au niveau de la santé publique). Au niveau national, c'est probablement le facteur le plus important (le plus grand vendeur autour du monde est en Chine, leader mondial dans le domaine).
\end{itemize}
 Aujourd'hui, les PHEV (hybdides) couvrent les 40km de déplacement journaliers demandés en suisse, mais restent très onéreux. Les BEV (totalement électriques) représentent tout de même 2/3 de la part totale de marché (à vérifier???). Aujourd'hui, le véhicule électrique représente globalement 5\% de la part du marché automobile. La Chine a doublé cette part du marché chaque année depuis environ 2010. \par
 La question aujourd'hui semble donc ne plus être la survie de ces véhicules sur le marché, mais plutôt la façon dont ils vont se propager. 
 \begin{enumerate}
 	\item Drivers for OEMs (fleet CO2 target values, strategic positioning)
 	\item Policy driver (Decarbonisation, pollution reduction, Energy efficiency, energy security, Balance of payment)
 	\item Drivers for owners (ecologic awareness, appeal of the new, vehicle performances, TCO?)
 \end{enumerate}
 Les véhicules électriques tiennent-ils leurs promesses? 
\begin{enumerate}
	\item PHEV/BEV (hybride/électrique) 	
 	\item Les constructeurs veulent absolument aller vers l'hybride, car cela leur permet de continuer à produire des composants qu'ils savent déjà faire.
 	\item Les biocarburants sont en concurrence directe avec l'agriculture nourricière, et ils ont un potentiel de décarbonisation faible, car les émissions de CO2 à la consommation sont largement déséquilibrés par l'énergie grise à la production. Pour l'instant, rien de convaincant (peut-être la génération 3 des algues et synthétiques?)
 	\item Hydrogène (piles à combustible) ? pas convaincant apparemment, mais déjà sur le marché. Peut-être?
\end{enumerate}

\textbf{L'électrification ne va pas de soi!} L'empreinte carbone des véhicules électriques n'est pas nulle: émissions lors de la génération d'électricité et de la manufacture de batterie (en Chine, où on a 80\% de charbon, c'est pas gagné). En suisse, on est à peu près à 160g, donc c'est actuellement pertinent de rouler à l'électrique. En Chine, rouler électrique n'est pas pertinenent d'un point de vue purement CO2, à cause de la pollution due au charbon, mais dans l'idée de sanité des centre-villes, cela fait du sens. L'important, partout autour du monde, est d'abaisser l'empreinte CO2 de la fabrication de batteries. \par
 En moyenne, un moteur thermique est autour de 18\% de rendement (dans conditions réalistes de start-stop, donc). Pour l'électrique, on est à une efficacité de 29\%. Ces deux chiffres sont pris par rapport à toute la chaîne (extraction de carburant, distribution du carburant, stockage, efficacité du moteur...). Les comparaisons 20\% contre 90\% c'est donc du bullshit. Pour ce qui est du prix, les companies de manufacture de voitures thermiques calculent leurs intérêts et paient les amendes plutôt que le développement si les sanctions ne sont pas assez élevées. Les véhicules thermiques et hybrides ont la particularité de nécessiter beaucoup plus de coûts post-achat, comparé aux voitures électriques dont les coûts post-achat sont presques nuls. Ils reviennent donc moins cher sur le long terme. Cependant, il faut penser que les taxes sur les carburants minéraux, qui représentent une certaine source de revenu pour l'(??????? état?) au niveau des taxes, et il faut s'attendre à ce qu'(ils) ne laissent pas passer ce revenu lors de la migration vers l'électrique. \par
 On peut penser à une synergie entre batterie électrique de la maison et celle de la voiture (qui reste parquée une bonne partie du temps).

\section{19 mars 2019 \quad Energies: Approche des Sciences Sociales et Humaines}
Le but de cette séance est de donner un survol des questions non-techniques des questions énergétiques. \par
\textbf{Écologie Industrielle}: l'écologie = étude scientifique des écosystèmes, Industriel = ensemble des activités humaines dans la société technologique moderne. Ensemble, cela veut dire appliquer l'écologie à l'entier du système économique dit "industriel", qui est celui dans lequel on vit. \par
 Aujourd'hui, le système industriel est en pleine expansion. Cela s'illustre dans la quantité de matière et d'énergie utilisée à l'échelle globale. On voit également le début de l'"industrie 4.0" - terme marketing douteux: numérisation, intelligence artificielle, convergence NBIC (?). L'application de l'écologie est particulièrement importante dans ce système. \par
 Le message à retirer de cette discussion est:
 \begin{center}
 	Les enjeux énergétiques dépassent les questions techniques; la quasi-totalité des sciences sociales et humaines sont concernées par les enjeux énergétiques, tant l'énergie est une ressource importante aujourd'hui. \par
 	Les sciences sociales et humaines jouent un rôle croissant dans l'évolution du système énergétique; les méthodologies qualitatives ont aussi leur righeur et leur pertinence, en complément des méthodes quantitatives. \par
 	\textbf{Essentiellement}, il y a un grand intérêt à intégrer les paramètres non-techniques dans les projets d'ingénieurie.
 \end{center}
 \textbf{L'abondance énergétique} est l'élément clé qui rend possible la société industrielle: l'individu n'a plus besoin de se soucier de sa propre survie (dans la société industrielle occidentale bien sûr). On peut lier le début de cette abondance à l'invention de la machine à vapeur de James Watt (1781), qui a permit de pomper en grande quantité de l'eau, ce qui a permit l'extraction de quantités inimaginables de charbon $\rightarrow$ acier $\rightarrow$ machines... (Georgescu-Roegen\footnote{à lire: \emph{La Décroissance}. Dans le même thème, \emph{La Biosphère de l'Anthropocène} de Jacques Grinevald.}, pionnier de la décroissance: "une invention prométhéenne"). \par
 Ceci a donné naissance notamment à un domaine confirmé: l'histoire de l'énergie. Mathieu Arnoux a montré que la nécessité d'extraire de l'eau dans certaines zones est probablement la source de la disparition des systèmes féodaux en faveur des systèmes d'accords entre "pays". On a également noté que le tourisme (lustre électrique de l'hôtel Beau-Rivage) a grandement propulsé l'électrification de l'arc lémanique. Autour de 1918-1939, on pouvait voir des mines de charbon et puits de pétrole à Oron-La-Ville et Servion, respectivement. \par
 Pour revenir à l'importance croissante des SHS dans la thématique énergétique, on peut noter que de nombreuses revues sur le thème de l'énergie, considéré du point de vue des sciences humaines et sociales, sont également apparues dans la dernière décennie. Elles traitent notamment de problématiques comme l'acceptation sociale des nouvelles sources d'énergie. Ici, l'enjeu appartient à la sociologie. Une autre question intéressante, considérée notamment par des archéologues et sémiologues, est celle du marquage à long terme des zones de stockage des déchets nucléaire: comment avertir les générations du futur qu'il ne faut pas construire de villes sur ces zones? \par
 On touche ici à un autre domaine en développement: celui de la psychologie sociale appliquée aux questions de durabilité. On a par exemple le \emph{story telling} (communication narrative), dont l'idée est de partir du postulat que la majorité n'est pas intéressée/capable de s'intéresser à la transition énergétique, et qu'il faut donc les attaquer aux tripes. On peut également noter les domaines de la communication engageante, la dissonance cognitive, les \emph{nudge} ("coups de coude") (il y a plus de chance que quelqu'un installe des panneaux solaires sur son toit si son voisin en possède également), la "réactance" (une personne normalement constituée finit par resister activement face à des messages moralisateurs ou culpabilisateurs (\emph{backfire effect})), le \emph{moral disengagement} ("moi je mange moins viande, mais du coup embêtez-moi pas pour mes voyages en avion"), etc. Un exemple de communication narrative lié à la question esthétique est celle de l'installation d'éoliennes. L'idée générale rassemblant ces trucs est celle de la prioritisation des sentiments sur les idées rationnelles (!). \par
 Mais tout ceci n'implique-t-il pas un risque de manipulation important? C'est ce qu'on nomme les enjeux de "gouvernementalité". La gouvernementalité est l'analyse des rationalités des pratiques de gouvernement, en tant que "coduite des conduites". La gouvernementalité est une manière "douce" d'influencer les comportements, d'exercer un pouvoir diffus sur les populations (pourquoi nous brossons-nous les dents tous les jours, nous douchons-nous? -> gnégné idéalisme de propreté je sais plus le nom). Des sociologues ont montré qu'il y a de tels facteurs positifs et négatifs influençant le recyclage des batteries de véhicules électriques (A COMPLETER). \par
 Ce qui intéresse de plus en plus les producteurs et les distributeurs d'énergie, notamment l'électricité, est le comportement des consommateurs. Ces companies engagent depuis quelque temps des sociologues, anthropologues, etc. pour étudier comment les gens (A COMPLETER). Le genre est un domaine qui a explosé dans les dernières années: question de la spécificité homme-femme. S'il existe bien un domaine où il existe des barrières de genre très claires, c'est celui de l'énergie (femmes qui transportent l'eau en Afrique p. ex). (DETAILS SUR PDF MOODLE).\par
 L'immense majorité se contrefout de l'énergie en tant que telle. Ce qui intéresse les gens, ce sont les services énergétiques: qu'est-ce qu'on peut faire avec l'énergie (communiquer, transporter, être en bonne santé...). Par exemple, le genre de questions qu'on se pose est:
 \begin{itemize}
 	\item Comment et dans quel but les personnes utilisent de l'énergie / différents types d'énergies?
 	\item Quelles sont les différentes conceptions culturelles de l'énergie et de son utilisation?
 	\item Comment ces concetpions (par exemple les conventions, les normes liées au confort, à la propreté etc.) changent-elles dans le temps? (aujourd'hui on est en T-shirt à la maison, il y a un siècle on se couvrait chez soi)
 	\item Qu'est-ce qui est considéré comme un niveau de vie "décent" et quelle quantité d'énergie est-elle nécessaire pour l'atteindre?
 	\item Comment les personnes exploitent-elles l'énergie à la fois quantitativement et qualitativement pour "donner du sens" à leur monde?
 	\item Quels sont les liens entre les transformations socio-culturelles et la notion de progrès?
 \end{itemize}
En suisse, dans le cadre de la stratégie énergétique 2050 et sur mandat du Conseil fédéral, le Fonds national a lancé deux Programmes Nationaux de Recherche (PNR): PNR 70 ("Virage énergétique", aspects techniques) et PNR 71 ("Gérer la consommation d'énergie", aspects sociaux, économiques, etc.).

\subsection{Trois principaux types de théories sociologiques}
Les approches diffèrent selon la manière de concevoir la nature des interactions sociales. Il existe trois conceptions principales:
\begin{itemize}
	\item Les théories orientées sur la \textbf{raison individuelle}: l'organisation de la société résulte de l'action cumulée d'individus, agissant indépendamment, se comportant de manière rationnelle, pour satisfaire leurs intérêts\footnote{Les idées fondatrices du modèle capitaliste se situent dans cette conception.}. Échelle micro, concernant l'action hunmaine.
	\item Théories orientées sur l'\textbf{action collective}: l'action humaine est contrainte et reflète les structures collectives. Échelle macro, concerne les normes, valeurs et institutions sociales.
	\item Théories orientées sur les \textbf{oppositions "individus" contre "structures"}. Troisème voie, médiane et unificatrice. Théories orientées sur la \emph{culture}, les structures symboliques et collectives de la connaissance et les pratiques sociales. Échelle méso. 
\end{itemize}
 Le paradigme dominant de la consommmation dans le système actuel est celui de la responsabilité individuelle\footnote{On peut noter l'exemple de la campagne écologique suisse "Battery-Man", qui encourageait chacun à rapporter ses piles usagées. Le symbole même du super-héro comme représentation de chacun représente cette confiance (aveugle) en l'individu.}. On a remarqué que de mettre ainsi un poids sur l'individu ne fonctionne que très mal. Il paraît dès lors plus efficace de cibler les pratiques sociales, c'est-à-dire des pratiques que l'on a apprises (de ses parents, de ses connaissances et amis...) plutôt que la responsabilité individuelle, afin de pouvoir franchir le gouffre de l'inaction que l'on nomme parfois \emph{value-action gap} ou \emph{information-action gap}. \par
 Afin de savoir cibler ces pratiques sociales, il est impératif de savoir ce qu'elles sont. La gestion du linge sale est un exemple d'activité qui peut être considérée en tant que pratique sociale: aujourd'hui, on jette indifféremment un vêtement à la lessive après l'avoir porté une fois\footnote{Dans le cadre de la lessive, le "jeans challenge" proposait aux participants de porter le même jeans pendant 4 semaines sans le laver, afin de faire réaliser aux gens qu'il est parfaitement possible (et c'est là même l'utilité du jeans) de porter une paire de pantalons sans les laver tous les trois jours.}. Il apparaît également que beaucoup de gens ne veulent pas utiliser le bouton "économique" de leur machine à lessive, car ils croient que la qualité du lavage sera moins bonne: il s'agit là d'un fort problème de communication. \par
 Maintenant qu'on sait où viser pour modifier les pratiques écologiquement problématiques, il faut s'interroger sur la \emph{manière} dont on va procéder. On peut imaginer trois stratégies:
 \begin{itemize}
 	\item \textbf{Reconfiguer une partie ou l'entier des éléments qui configurent des pratiques}. Par exemple, pour ce qui est de la pratique "conduire une voiture", on pourrait imaginer l'introduction de cours d'éco-conduite, des moteurs qui optimisent automatiquement la quantité d'énergie dépensée ou la synchronisation des feux de signalisation afin de perdre moins d'énergie dans les démarrages et arrêts répétitifs en fluidifiant le trafic.
 	\item \textbf{Substituer une pratique cible par une pratique alternative}. Pour reprendre l'exemple de la mobilité, on pourrait développer un environnement qui encourage le transfert vers les transports publics, l'écomobilité (mobilité hybride) ou la mobilité "douce" (déplacement à pied ou à vélo).
 	\item \textbf{Reconfigurer des complexes de pratiques}. Ici, l'idée est de cibler non pas une mais plusieurs pratiques entrelacées. Un exemple concret est celui d'Oklahoma City, ville américaine dont le maire a lancé à sa population de perdre  communément un million de livres\footnote{Environ 450'000 kg.}de graisse. Il n'est pas resté passif, et a mis en place un programme de réaménagements urbains importants afin de permettre à la population de se prêter au défi collectif\footnote{www.thiscityisgoingonadiet.com}. Un autre exemple concret, dans le cadre de la promotion des produits locaux et de saison, est le Restaurant "Les Mangeurs", à Genève: plutôt que d'encourager chacun à faire des efforts individuels quant à leur consommation alimentaire, ils ont organisés des ateliers afin d'apprendre à toustes comment préparer topinambours, salsifis, et autres produits locaux et souvent négligés.
 \end{itemize}

% COMPLETER: A PARTIR D'ICI AINSI QUE LE DEBUT
\subsection{Rôle structurant des technologies / pratiques de la communication et de l'information (ICT)}
 La plupart des pratiques passent désormais par Internet (COMPLETER). Un aspect intéressant est celui de l'articulation de la technologie avec les pratiques. \par
 Émergence de nouveaux modèles d'affaires: le déclin de la "rigidité de l'offre" ("mes clients peuvent choisir la couleur de leur voiture, pour autant qu'elle soit noire" - Ford). L'idée est qu'on va plutôt vendre la performance, le résultat que la matière (voir idée du \emph{chemical leasing}, acheter de la "surface peinte" plutôt que de la peinture (vendre la performance plutôt que la matière première), ou exemple de la chaudière dont on paye le confort thermique, mais pas la propriété (le propriétaire, c'est à dire le producteur, veut donc la faire durer aussi longtemps que possible)). Vendre une performance et de la satisfaction plutôt que de la matière est un \emph{business system} émergent (PSS). \par
 Si l'on veut résumer par un mot toutes les branches de la "durabilité", on pourrait utiliser le terme "découpler": garder/augmenter le confort en utilisant moins. La clé, selon l'ingénieur, est l'\textbf{efficacité}. Mais l'efficacité est-elle efficace?
 \begin{itemize}
 	\item Plus d'output par unité d'input
 	\item Moins d'input par unit d'output
 \end{itemize}
 Faire augmenter l'efficacité ne mène pas, contrairement à ce que beaucoup pensent, à une baisse de la consommation: c'est même inverse! Lorsqu'on vend des voitures hybrides, les gens déculpabilisent et plus de personnes vont conduire, et conduire plus. Autre exemple: le E-Commerce. Les livraisons rapides ont fait augmenter les embouteillages plutôt que de les faires baisser. \par
 Définition de l'effet rebond:
 \begin{itemize}
 	\item Avec un prix d'énergie fixe, les gains en efficacité énergétique vont augmenter la consommation (A COMPLETER)
 \end{itemize}
 Le paradoxe de Jevons: "It is wholly a confusion of ideas to suppose that the exonomical use of fuel is equivalent to a diminished consumption. The very contrary is the truth. As a new rule, ..... A COMPLETER". Ex: on met chez nous des ampoules qui consomment moins. Cela coûte moins cher, donc on en achète plus, on les laisse plus allumées. Ou: on achète une pompe à chaleur, donc on chauffe plus et plus longtemps. Bref, c'est assez évident quand on y pense, l'effet rebond. \par
 Une question hérétique vient ici à l'esprit: ne faudrait-il donc pas accroître l'efficacité? Le hummer: "Big is the new small." S'il y avait uniquement des voitures qui consomment plus de 60l/100km, moins de personnes utiliseraient-elles moins la voiture? (ce n'est probablement pas une bonne idée, mais plutôt)
 Il ne faut pas renoncer à l'efficacité, mais il ne faut pas non plus ce voiler la face: il est important d'arrêter de croire que l'efficacité technique seule peut améliorer la situation. Elle doit être accompagnée de mesures sociales, économiques, politiques, etc (surtout sociales). \par
 À voir: l'échelle de Kardashev (sur wikipedia):
 \begin{enumerate}
 	\item A COMPLETER trois types de civilisations
 	\item 
 	\item 
 \end{enumerate}
Rapport spécial du GIEC: publié en octobre 2018. A proposé plusieurs approches pour maintenir le réchauffement global en-dessous de 1.5 degrés Celsius:
\begin{enumerate}
	\item Carbon Dioxyde Removal (CDR): capture and storage (CCS), capture and utilization (CCU). CCS: Problème d'acceptation du public: lac Nyos (Cameroun). CCU: valoriser le CO2 avec: réseaux urbains pour thermie urbaine, chimie (recherches en cours, déjà prometteuses), voir site de CO$_2$ Value Europe.
	\item Solar Radiation Management (SRM)
\end{enumerate}

Questions délicates:
\begin{itemize}
	\item À qui appartient le CO$_2$ anthropogénique atmosphérique? Il n'a aujourd'hui pas de statut juridique, n'étant pas une "chose" (condition nécessaire à avoir un tel statut).
	\item Acceptabilité sociale du CCU
	\item Selon quel \emph{business model} l'exploiter?
	\item Comment distribuer la richesse carbonée globale?
	\item Comment éviter une surexploitation?
	\item Une stratégie pour le développement régional?
	\item Enjeux éthiques (responsabilités)
\end{itemize}

 (note sans rapport: www.climatechangenews.com) \par
 Symbioses ou synergies (éco-)industrielles: faire des optimisations au niveau du système; le déchet d'une entreprise devient la matière première d'une autre. Ceci demande une analyse de tous les flux dans un système d'entreprises, qui doivent se mettre ensemble pour profiter mutuellement. Ex: site industriel de Dunkerque, ou encore Kalundborg, au Danemark.


\section{26 mars 2019 \quad Carbon Capture Utilisation and Storage}
 Ce cours est basé sur le livre \emph{Introduction to Carbon Capture and Sequestration}. \par
 Nous commencerons avec ``le bon'': le cycle naturel du CO$_2$. Avec une loi de thermodynamique, on peut calculer facilement la température moyenne de la Terre si elle n'avait pas d'atmosphère: elle est de 6 degrés Celsius. Cependant, ce calcul ne prend pas en compte l'effet de serre, ni l'énergie que l'on utilise en tant qu'humains et que l'on convertit en tant que chaleur. \par
 Ce dernier minime représente une quantité minime de la chaleur de la Terre: moins d'un millième de celle apportée par le soleil. Nous pouvons donc conclure que les émissions directes de chaleur causées par notre consommation en énergie sont négligeables. \par
 Si l'on prend en compte l'effet albedo\ldots COMPLETER \par
 L'atmosphère empêche quasi-totalement les rayons UV émis par le soleil d'atteindre la surface de la terre. La chaleur infrarouge (``earthshine'') émis par la Terre est gardée à l'intérieur de l'atmosphère à cause de l'effet de serre. En 1986, un papier scientifique prédisait que si la concentration en CO$_2$ de l'atmosphère double, la température terrestre moyenne augmenterait de 5 degrés Celsius. Des estimations plus récentes prévoyent une augmentation de 2.2 degrés pour le même changement de concentration, en prenant en compte plus de facteurs. \par
 Des données centralisées pour le niveau et la température de la mer n'existent que depuis peu. Pour des données plus anciennes, on a du consulter différents documents historiques. Les conclusions sont claires: dans les dernières décénnies, le niveau et la température de la mer montent plus vite, et les glaciers se retirent. \par
 Les causes de ce réchauffement global sont principalement dues aux activités humaines: bien que les fluctuations de l'activité solaire causent un petit changement, mais le bilan penche en (dé)faveur de l'humain, les émissions de CO$_2$ en tête. \par
 Le dioxyde de carbone n'est pas la forme la plus stable du carbone. La forme la plus stable est le carbonate (qui est un solide), mais la réaction pour passer du carbone au carbonate est très lente. La quantité de carbone sous forme de carbonate permet de mesurer la quantité de carbone dans l'air à différentes époques dans le passé. Il faut savoir que la concentration de CO$_2$ varie grandement même à travers une journée: au niveau du sol (1--6 mètres), elle est beaucoup plus élevée pendant la nuit que pendant le jour, à cause de la photosynthèse (cette variation est bien plus faible un peu plus haut, déjà à une vingtaine de mètres au-dessus du sol). Cette fluctuation est aussi présente à l'échelle d'une année: en été, la concentration est plus faible qu'en hiver. La concentration en CO$_2$ de l'atmosphère était bien plus élevée il y a plusieurs milliard d'années, et n'a fait, sur le long-terme, que de chuter depuis. La température du soleil a cependant augmenté depuis. \par
 Cycle du CO$_2$: le carbone atmosphérique acidifie les pluies, qui lui permettent de se déposer dans le sol. Il passe en mode carbonate et, par les cours d'eaux, il se retrouve dans la mer. Là, il est utilisé comme coquille par des planctons, qui finissent par mourir et se déposer au fond de l'océan. Mais le carbonate ne reste pas au fond: par les mouvements des plaques tectoniques, il se retrouve éjecté dans l'atmosphère à nouveau par les volcans, qui émettent environ 0.15 Gt de carbones par an. \par
 Ce cycle est équilibré, mais prend beaucoup de temps à se réguler: \textbf{le véritable problème des émissions de CO$_2$ causées par les humains n'est pas leur ampleur, mais la durée sur laquelle elles prennent place}. 45\% de nos émissions carbon restent dans l'atmosphère, 25\% se retrouve dans l'océan de surface et 30\% ``dans la terre''. \par



\section*{02 avril 2019 \quad marché énergétique}
	Jusqu'à 1850, la seule source d'énergie était la biomasse. À partir de l'arrivée du charbon, la part d'énergie fossile n'a cessé d'augmenter jusqu'à aujourd'hui (2018 environ). \par
	Le système actuel est à 80\% fossile. Ce système aurait pu continuer si l'on avait pas atteint certaines limites: le pic pétrolier, \dots Le pic pétrolier du \emph{crude oil} est passé à l'échelle globale. Toutes les prédictions quant aux prochaines décennies s'accordent à dire que les demandes en énergie vont continuer à croître (principalement en Inde et Chine). Il y a une corrélation forte entre population et demande énergétique, mais il semble que la demande soit beaucoup plus fortement liée au PIB du pays: la consommation est donc indubitablement liée à la croissance économique. En conséquence de cette croissance économique, la demande d'énergie moyenne par personne dans le monde augmente. \par
	Une demande qui augmente alors que les réserves sont finies implique que l'\emph{energy gap} va être énorme. Aujourd'hui, la pénurie de pétrole existe déjà dans certains pays (par exemple au Malawi, stations service vides). Évidemment, les pays les plus pauvres souffrent le plus. Même en Suisse, il est arrivé que nous soyons proche de la limite (A INVESTIGUER). \par
	Quelles sont les stratégies pour pallier à ces pénuries dans le futur?
	\begin{itemize}
		\item \textbf{La sécurité énergétique par diversification}: ne pas être dépendant d'une unique source d'énergie (voir épisode de la crise Khadafi) (4 mois de réserve obligatoires en Suisse). On va de plus en plus dépendre du gaz russe car les gaz norvégiens (et autres) s'épuisent. \par
		Les prédictions pour 2035 montrent que l'Inde, la Chine et l'UE vont tous augmenter leur dépendance à l'étranger (pas les USA, qui sont assez suffisants). Cela engendre une peur quant à cette dépendance. \par
		Depuis 2017, les USA sont déjà des exportateurs nets de gaz, et seront probablement exportateurs de pétrole autour de 2027. 
	\end{itemize}
	Comment couvrir le \emph{energy gap}?
	\begin{enumerate}
		\item \textbf{Fuels non-conventionnels}: ce sont des ressources fossiles qui ont historiquement été impossible à produire pour des raisons économique ou techniques. On peut nommer le gaz de schiste (\emph{shale oil}), le \emph{kerogen shale}, ou les sables bitumineux (notamment en extraction au Canada, qui se trouvent proche de la surface mais demandent un important raffinement). Les USA dépendent déjà lourdement des fuels non-conventionnels: le pic n'a pas été atteint, contrairement au pic du \emph{crude oil}. On estime qu'encore 99\% des énergies fossibles sont encore intouchées sur Terre, mais dans la vaste majorité inaccessible.
		\item \textbf{Énergies renouvelables}:
		\item \textbf{Efficacité énergétique}: la façon dont on consomme actuellement est particulièrement inefficace (une voiture thermique est à environ 18\% de rendement sur toute la chaîne, par exemple). 
		\item \textbf{Découpler la croissance économique de la demande énergétique}: peut-on le faire? Une étude de l'EPFL montre que même si la Suisse paraît augmenter son PIB tout en ayant une demande énergétique décroissante, il se trouve que l'énergie grise des produits transformés importés a augmenté (doublé entre 2001 et 2011) avec la quantité de ces produits. La Suisse s'est ``tertiarisé'' et s'est désindustrialisée, ce qui donne cette apparence de diminution de la demande, mais cette dernière s'est donc seulement déplacée. Il semble que l'on ne puisse donc pas affirmer qu'il est possible de découpler la croissance économique de la demande énergétique.\par
		Si les prix du fossile restent bas (ce qui est l'inverse des prévisions):
		\begin{itemize}
			\item énergies renouvelables ne sont pas compétitives
			\item les inconventionnels non-plus
			\item mais prospection faible
			\item et EOR faible
		\end{itemize}
		Si ils augmentent:
		\begin{itemize}
			\item énergies renouvelables compétitives
			\item A COMPLETER \dots
		\end{itemize}
		Les impacts sur l'environnement de l'extraction des énergies non-conventionnelles sont beaucoup plus hautes: eau pour le \emph{fracking}, moteurs pour transporter sables bitumineux, déforestation et pollution des sols, etc (la déforestation des forêts boréales au Canada sont énormes, à l'échelle de l'Amazonie). \par
		On assiste à un désinvestissement massif (6'000 milliards par 800 des plus grandes institutions) dans les énergies fossiles. Seul le gaz reste d'intérêt (la plupart des compagnies pétrolières sont en fait maintenant des compagnies gazières). \par
		On prévoit que les émissions CO$_2$ de la Chine vont se stabiliser autour de 2025. Actuellement, on prévoit que si tous les pays des accords font tout ce qu'ils ont prévu, le réchauffement climatique a 90\% de chance d'atteindre 3.7$^\circ$C. \par
		Il ne faut pas oublier que beaucoup d'autres problèmes existent: le changement climatique est moins dangereux actuellement que l'effondrement de la biodiversité, par exemple.
		\item \textbf{Le nucléaire}: de plus en plus cher, et de nombreux désavantages, mais peu visible et empeinte carbone quasi-neutre (meilleur que photovoltaïque)
		\item \textbf{La décroissance}: nous ne l'aborderons pas car il sort du cadre du cours, mais très intéressant.
	\end{enumerate}



















\end{document}
