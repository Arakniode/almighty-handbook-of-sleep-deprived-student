\documentclass{article}

\title{Algèbre Linéaire 1 - Scherer}
\author{Benjamin Bovey -  EPFL IC}
\date{Année 2018-2019}

\usepackage[utf8]{inputenc}
\usepackage{amssymb}
\usepackage{mathtools}
\usepackage{hyperref}

\usepackage{geometry}
\geometry{left=1.5cm, right=1.5cm, top=1.5cm, bottom=1.5cm}

\numberwithin{equation}{section}

\newcommand\defn{\vspace{4px}\noindent \quad {\large \underline{\textsc{Définition }}: }}
\newcommand\thm[1]{\vspace{4px}\noindent \quad {\large \underline{\textsc{Théorème #1}}: }}
\newcommand\meth[1]{\vspace{4px}\noindent \quad {\large \underline{\textsc{Méthode : #1}} }}

\providecommand{\abs}[1]{\lvert#1\rvert} 
\providecommand{\norm}[1]{\lVert#1\rVert}

% TODO check if we can override \Im and \Ker
\DeclareMathOperator{\rang}{rang}
\DeclareMathOperator{\im}{Im}
\DeclareMathOperator{\ke}{Ker}
\DeclareMathOperator{\lign}{Lign}
\DeclareMathOperator{\mult}{mult}
\DeclareMathOperator{\proj}{proj}
\DeclareMathOperator{\vect}{Vect}

%---------------------------------------------------------------------------------------------------------------------

\begin{document}

\maketitle

%---------------------------------------------------------------------------------------------------------------------

\section*{Introduction}
Ce document est destiné à résumer les cours d'algèbre linéaire 1 donnés par Mr. Jérôme Scherer. Pour l'instant il regroupe la matière à partir du cours 16. Voici le \href{https://github.com/Arakniode/almighty-handbook-of-sleep-deprived-student}{GitHub du projet}.

%---------------------------------------------------------------------------------------------------------------------

\section{Inversibilité}
Les propositions suivantes sont équivalentes:
\begin{itemize}
	\item La matrice \(A\) est inversible
	\item L'application représentée par \(A\) est bijective (\(\Rightarrow\) injective et surjective)
	\item Les colonnes de \(A\) forment une base de \(\mathbb{R}^n\)
	\item \(\im(A) = \mathbb{R}^n\)
	\item \(\dim \im(A) = n\)
	\item \(\rang(A) = n\)
	\item \(\ke(A) = \{0\}\)
	\item \(\dim \ke(A) = 0\)
\end{itemize}

%---------------------------------------------------------------------------------------------------------------------

\section{Vecteurs propres et valeurs propres}
Les propositions suivantes sont équivalentes:
\begin{itemize}
	\item \(0\) est valeur propre de \(A\)
	\item \(\ke(A) \neq 0\)
	\item \(\rang(A) < n\)
	\item \(A\) n'est pas surjective
	\item \(A\) n'est pas inversible
\end{itemize}
Les valeurs propres d'une matrice \textbf{triangulaire} sont les coefficients diagonaux de la matrice. Cette propriété tient donc bien sûr pour les matrices \textbf{diagonales}, qui sont des cas particuliers de matrices triangulaires.
\subsection{Le polynôme caractéristique}
Le polynôme caractéristique d'une matrice n'existe que pour les matrices carrées, car le déterminant est uniquement défini sur les matrices carrées. \\
Soit \(A\) une matrice \(n \times n\), et soit \(\chi_A(\lambda)\) son polynôme caractéristique. Alors
\begin{equation}
	\chi_A(\lambda) = \det(A-\lambda I_n)
\end{equation}
Une valeur propre de \(A\) est une racine du polynôme caractéristique \(\chi_A(\lambda)\). 

\subsection{Espace propre associé à une valeur propre}
Soit \(\lambda\) une valeur propre de \(A\). Alors l'espace propre associé à \(\lambda\) est
\begin{equation}
	\ke(A-\lambda I_n)
\end{equation}

\subsection{Similitude}
\defn
\begin{center}
	\emph{Deux matrices carrées de taille \(n \times n\) sont \textbf{semblables} \\ s'il existe une matrice inversible \(P\) de taille \(n \times n\) \\ telle que \(A = P^{-1}BP\).}
\end{center}

En gros, deux matrices sont semblables si elles représentent la même application exprimée dans deux bases différentes. \\
Ce qui est important, c'est que:
\begin{center}
	\emph{Deux matrices semblables ont le même polynôme caractéristique, \\ et donc les mêmes valeurs propres.}
\end{center}
\textbf{Attention:} le fait que deux matrices aient les mêmes valeurs propres n'implique pas qu'elles sont semblables.

\subsection{Multiplicité des valeurs propres}
On fait la différence entre la multiplicité \textbf{algébrique} d'une valeur propre et sa multiplicité \textbf{géométrique}.
\textbf{Définition:}
\begin{center}
	La \textbf{multiplicité algébrique} d'une valeur propre \\ est sa multiplicité en tant que racine de \(\chi_A(\lambda)\). \\ \vspace{10pt}
	La \textbf{multiplicité géométrique} d'une valeur propre \\ est la dimension de l'espace propre qui lui est associé.
\end{center}
On écrira d'ailleurs \(\mult(\lambda)\) pour la multiplicité algébrique de \(\lambda\) et \(\dim(E_\lambda)\) pour la multiplicité géométrique de \(\lambda\).

\subsection{Diagonalisabilité}
\thm{} une matrice $A$ de taille $n$ est diagonalisable si et seulement si:
\begin{itemize}
	\item \(\chi_A(\lambda)\) est scindé
	\item \(\forall \lambda\), on a \(\dim(E_\lambda) = \mult(\lambda)\)
\end{itemize}
Autrement dit:
\begin{center}
	Une matrice \(A\) est diagonalisable si et seulement si \\ la somme des multiplicités géométriques de ses valeurs propres \\ est égale à \(n\).
\end{center}
Une condition suffisante \emph{mais pas nécessaire} est la suivante:
\begin{center}
	Une matrice \(A\) est diagonalisable \\ si elle possède \(n\) valeurs propres \underline{distinctes}.
\end{center}

%---------------------------------------------------------------------------------------------------------------------

\section{Orthogonalité}

\subsection{Idées générales}

Soit $A$ une matrice de taille $n \times m$. Alors $A^T$ est de taille $n \times m$.
\begin{itemize}
	\item Les lignes de $A^t$ sont les colonnes $\vec a_i$ de $A$
	\item $A^T\vec x = \vec 0 \iff \vec x \perp \im A$
	\item Les coefficients $(A^TA)_{ij}$ sont les produits scalaires $\vec a_i \cdot \vec a_j$
	\item Les colonnes de $A$ sont orthogonales $\iff A^TA$ est diagonale
	\item Les coefficients de $AA^T$ sont les produits scalaires des lignes de $A$
	\item Les lignes de A sont orthogonales $\iff AA^T$ est diagonale
\end{itemize}

\begin{align}
\shortintertext{Soit \(A\) une matrice de taille \(m \times n\):}
	&\ke A = (\lign A)^\perp \\
	&\ke A^T = 	(\im A)^\perp \\
\shortintertext{Preuve:}
	\nonumber &A \cdot \vec x \iff \vec x \perp \text{chaque ligne de } A \\
	\nonumber &\im A = \lign A^T
\end{align}

\subsection{Base orthogonale}
Soit \(W\) un sous-espace de \(\mathbb{R}^n\) et \((\vec u_1, \dots, \vec u_k)\) une base \emph{orthogonale} de \(W\). Alors
\begin{align}
	\vec w \in W = \alpha_1\vec u_1 + \dots + \alpha_k\vec u_k, \\
\intertext{et on calcule les $\alpha_j$ de la manière suivante:}
	\label{eq:orthoproj} \alpha_j = \dfrac{\vec w \cdot \vec u_j}{\norm{\vec u_j}^2}
\end{align}

\subsection{Matrice orthogonale}
\thm{} matrices orthogonales \\
Les propositions suivantes sont équivalentes:
\begin{itemize}
	\item $U$ est orthogonale.
	\item Les colonnes et lignes d'une matrice \textbf{carrée} $U$ de taille $n$ sont \textbf{orthonormées}
	\item $U^TU = I_n$
	\item $U^{-1} = U^T$
\end{itemize}
Une matrice orthogonale représente une transformation linéaire qui préserve les distances et l'orthogonalité (isométrie, p. ex rotation ou symétrie). \\

\thm{} préservation des longueurs \\
Soit $U$ une matrice orthogonale. Alors:
\begin{enumerate}
	\item $\norm{U\vec x} = \norm{\vec x} \forall \vec x \in \mathbb{R}^n$
	\item $U\vec x \cdot U\vec y = \vec x \cdot \vec y$
	\item $U\vec x \perp U\vec y \iff \vec x \perp \vec y$
\end{enumerate}

\subsection{Projection orthogonale}
\thm{} projection d'un vecteur sur un sous-espace \\
Soit $(\vec u_1, \dots, \vec u_k)$ une base orthogonale de $W$, sous-espace de $\mathbb{R}^n$. 
\begin{center}
	$\forall \vec y \in \mathbb{R}^n, \vec y = \hat y + \vec z$
	où $\hat y \in W$ et $\vec z \in W^\perp$.
\end{center}
Cette décomposition est unique. On peut calculer $\hat y = \proj_W \vec y$ à l'aide de \ref{eq:orthoproj}. \\

La meilleure méthode pour construire la projection est la suivante:
\begin{enumerate}
	\item Vérifier que la base de $W$ est orthogonale (c'est la base de la base!)
	\item Calculer les normes au carré des vecteurs de base $\vec u_i$
	\item Calculer les produits scalaires $\vec y \cdot \vec u_i$
	\item Appliquer \ref{eq:orthoproj} pour chaque $u_i$
	\item Calculer $\vec z = \vec y - \hat y$ et vérifier que $\vec z \perp W$
\end{enumerate}

\thm{} projection d'un vecteur sur un sous-espace, cas d'une base orthonormée \\
\begin{center}
	Soit $U$ la matrice dont les colonnes sont les vecteurs $\vec u_1, \dots, \vec u_k$ \\ d'une base \emph{orthonormée} de $W$. Alors \\ $\boxed{ \proj_W \vec y = UU^T\vec y }$
\end{center}

\subsection{Méthode des moindres carrés}
\textbf{Idée}: On veut approximer une solution (appelée \emph{solution au sens des moindres carrés}) d'un système incompatible de la forme $A\vec x = \vec b$, où $A\vec x \in \im A$, mais $\vec b \notin \im A$. On va donc vouloir résoudre le système en remplaçant $\vec b$ par $\proj_{\im A} \vec b = \hat b$. Comme $\hat b \in \im A$, on obtient le système compatible $A\hat x = \hat b$ qui permet d'obtenir 'approximation la plus proche de la solution. \\
Cependant, on n'aimerait pas devoir calculer $\hat b$. On peut trouver une équation, appelée l'équation normale du système, qui permet de trouver la solution au sens des moindres carrés sans calculer $\hat b$.

\thm{} équation normale 
\begin{equation}
	\boxed{ A\hat x = \hat b \iff A^TA \hat x = A^T\vec b }
\end{equation}
On appelle $\hat x$ la \emph{solution au sens des moindres carrés} du système. \\

\textbf{Remarque}: Il existe généralement une infinité de solutions au sens des moindres carrés. \\ La solution est unique $\iff A$ est injective $\iff A$ est inversible.

\textbf{Remarque}: On appelle $\norm{\vec z} = \norm{\vec b - \hat b}$ l'\emph{erreur quadratique} ou \emph{écart quadratique}. Graphiquement, c'est la distance entre le vecteur $\vec b$ et la solution au sens des moindres carrés $\vec x$, donc la différence entre la solution "idéale" et la solution approximée.

\textbf{Méthode}:
\begin{enumerate}
	\item Calculer $A^TA$
	\item Calculer $A^T\vec b$
	\item Résoudre le système $A^TA\hat x = A^T\vec b$ pour obtenir la solution au sens des moindres carrés $\hat x$
	\item (Calculer $\norm{\vec b - \hat b}$ pour obtenir l'écart quadratique)
\end{enumerate}

\subsubsection{Méthode de Gram-Schmidt}
\textbf{But}: Trouver une base \emph{orthogonale ou orthonormée} d'un sous-espace $W$ de $\mathbb{R}^n$. \\
\textbf{Idée}: Partir d'une base quelconque, et utiliser les projections orthogonales pour créer une base orthonormée.

\subsection{Matrices symétriques}
\defn
\begin{center}
	Une matrice carrée $A$ est \emph{symétrique} $\iff A^T = A$, i.e. $a_{ij} = a_{ji}$.
\end{center}

\thm{}
\begin{center}
	Les espaces propres d'une matrice symétrique sont orthogonaux entre eux.
\end{center}

\defn
\begin{center}
	Une matrice carrée $A$ est diagonalisable par un changement de base orthonormée ou \emph{orthodiagonalisable} s'il existe une matrice $P$ orthogonale telle que $P^TAP$ est diagonale.
\end{center}

\thm{}
\begin{center}
	$A$ est orthodiagonalisable $\iff A$ est symétrique.
\end{center}

\thm{spectral}
\begin{center}
	Soit $A$ une matrice symétrique. Alors, $A$ \textbf{est orthodiagonalisable}.
\end{center}

On peut également faire les remarques suivantes sur $A$:
\begin{enumerate}
	\item $A$ admet $n$ valeurs propres réelles, compte tenu de leur multiplicité
	\item Pour toute valeur propre $\lambda$ on a $\mult(\lambda) = \dim E_\lambda$
	\item Les espaces propres de $A$ sont perpendiculaires 2 à 2 ($\iff$ si $\lambda \neq \mu$, alors $E_\lambda \perp E_\mu$)
\end{enumerate}

\meth{orthodiagonalisation}
\begin{enumerate}
	\item Vérifier que $A$ est symétrique
	\item Calculer $\chi_A(t)$ et en extraire les valeurs propres
	\item Calculer les espaces propres. Trouver une base orthonormée pour chaque espace propre par le procédé de Gram-Schmidt
	\item Assembler les bases orthonormées des espaces propres, on obtient une base orthonormée $B$ de $\mathbb{R}^n$
	\item La matrice $P$ dont les colonnes sont les vecteurs $\vec b_i$ de $B$ est orthogonale et $D = P^TAP$ est diagonale
\end{enumerate}
\textbf{Remarque}: comme $P$ est orthogonale, $P^{-1} = P^T$!

\subsection{Matrice de projection}
\defn 
\begin{center}
	Soit $\vec u$ un vecteur unitaire. Alors $A = \vec u \vec u^T$ est la matrice de la projection orthogonale sur $W = \vect(\vec u)$. \\ On a $A \vec x = \proj_{\vec u} \vec x$. 
\end{center}
\textbf{Remarque}: $\vec u$ est un vecteur propre de $A$ pour la valeur propre 1 car $A\vec u = (\vec u \vec u^T)\vec u = \vec u (\vec u^T \vec u) = \vec u (\vec u \cdot \vec u) = \vec u$. \\
\textbf{Remarque}: Si $W = \vect(\vec u)$, alors $W^\perp = \ke A$. \\
\textbf{Remarque}: Cette matrice de projection construite à partir d'un vecteur unitaire est un cas particulier de ce que nous avons vu avec les matrices de projection de la forme $UU^T$, où les colonnes de $U$ forment une base orthonormée.

\subsection{Décomposition spectrale}
\defn
\begin{center}
	L'ensemble des valeurs propres de $A$ est appelé \emph{spectre} de $A$.
\end{center}

\thm{}
\begin{align}
\shortintertext{Comme $D = U^TAU$,}
	A = UDU^T = \dots = \lambda_1 \vec u_1 \vec u_1^T + \overbrace{\dots}^{\text{cours 25}} + \lambda_n \vec u_n \vec u_n^T
\end{align}

\defn
\begin{center}
	$A = \lambda_1 \vec u_1 \vec u_1^T + \dots + \lambda_n \vec u_n \vec u_n^T$ est la \emph{décomposition spectrale} de la matrice symétrique $A$.
\end{center}
\textbf{Remarque}: c'est la somme de matrices contenant chacune 1 valeur propre dans sa diagonale. $A$ est décomposée en une combinaison linéaire de matrices de projection orthogonale!






\end{document}


