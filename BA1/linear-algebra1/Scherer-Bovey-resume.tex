\documentclass{article}

\title{Algèbre Linéaire 1 - Scherer}
\author{Benjamin Bovey -  EPFL IC}
\date{Année 2018-2019}

\usepackage[utf8]{inputenc}
\usepackage{amsmath}
\usepackage{amssymb}
\usepackage{mathtools}
\usepackage{hyperref}

\newcommand\defn{\textbf{Définition:}}
\newcommand\thm{\textbf{Théorème:}}

% TODO check if we can override \Im and \Ker
\DeclareMathOperator{\rang}{rang}
\DeclareMathOperator{\im}{Im}
\DeclareMathOperator{\ke}{Ker}
\DeclareMathOperator{\mult}{mult}

\begin{document}
\maketitle

\section*{Introduction}
Ce document est destiné à résumer les cours d'algèbre linéaire 1 donnés par Mr. Jérôme Scherer. Pour l'instant il regroupe la matière à partir du cours 16. Voici le \href{https://github.com/Arakniode/almighty-handbook-of-sleep-deprived-student}{GitHub du projet}.

\section{Inversibilité}
Les propositions suivantes sont équivalentes:
\begin{itemize}
	\item La matrice \(A\) est inversible
	\item L'application représentée par \(A\) est bijective (\(\Rightarrow\) injective et surjective)
	\item Les colonnes de \(A\) forment une base de \(\mathbb{R}^n\)
	\item \(\im(A) = \mathbb{R}^n\)
	\item \(\dim \im(A) = n\)
	\item \(\rang(A) = n\)
	\item \(\ke(A) = \{0\}\)
	\item \(\dim \ke(A) = 0\)
\end{itemize}

\section{Vecteurs propres et valeurs propres}
Les propositions suivantes sont équivalentes:
\begin{itemize}
	\item \(0\) est valeur propre de \(A\)
	\item \(\ke(A) \neq 0\)
	\item \(\rang(A) < n\)
	\item \(A\) n'est pas inversible
\end{itemize}
Les valeurs propres d'une matrice \textbf{triangulaire} sont les coefficients diagonaux de la matrice. Cette propriété tient donc bien sûr pour les matrices \textbf{diagonales}, qui sont des cas particuliers de matrices triangulaires.
\subsection{Le polynôme caractéristique}
Le polynôme caractéristique d'une matrice n'existe que pour les matrices carrées, car le déterminant est uniquement défini sur les matrices carrées. \\
Soit \(A\) une matrice \(n \times n\), et soit \(\chi_A(\lambda)\) son polynôme caractéristique. Alors
\begin{equation}
	\chi_A(\lambda) = \det(A-\lambda I_n)
\end{equation}
Une valeur propre de \(A\) est une racine du polynôme caractéristique \(\chi_A(\lambda)\). 

\subsection{Espace propre associé à une valeur propre}
Soit \(\lambda\) une valeur propre de \(A\). Alors l'espace propre associé à \(\lambda\) est
\begin{equation}
	\ke(A-\lambda I_n)
\end{equation}

\subsection{Similitude}
\textbf{Définition:}
\begin{center}
	\emph{Deux matrices carrées de taille \(n \times n\) sont \textbf{semblables} \\ s'il existe une matrice inversible \(P\) de taille \(n \times n\) \\ telle que \(A = P^{-1}BP\).}
\end{center}

En gros, deux matrices sont semblables si elles représentent la même application exprimée dans deux bases différentes. \\
Ce qui est important, c'est que:
\begin{center}
	\emph{Deux matrices semblables ont le même polynôme caractéristique, \\ et donc les mêmes valeurs propres.}
\end{center}
\textbf{Attention:} le fait que deux matrices aient les mêmes valeurs propres n'implique pas qu'elles sont semblables.

\subsection{Multiplicité des valeurs propres}
On fait la différence entre la multiplicité \textbf{algébrique} d'une valeur propre et sa multiplicité \textbf{géométrique}.
\textbf{Définition:}
\begin{center}
	La \textbf{multiplicité algébrique} d'une valeur propre \\ est sa multiplicité en tant que racine de \(\chi_A(\lambda)\). \\ \vspace{10pt}
	La \textbf{multiplicité géométrique} d'une valeur propre \\ est la dimension de l'espace propre qui lui est associé.
\end{center}
On écrira d'ailleurs \(\mult(\lambda)\) pour la multiplicité algébrique de \(\lambda\) et \(\dim(E_\lambda)\) pour la multiplicité géométrique de \(\lambda\).

\subsection{Diagonalisabilité}
\thm une matrice \(A\) est diagonalisable si et seulement si:
\begin{itemize}
	\item \(\chi_A(\lambda)\) est scindé
	\item \(\forall \lambda\), on a \(\dim(E_\lambda) = \mult(\lambda)\)
\end{itemize}
Autrement dit:
\begin{center}
	Une matrice \(A\) est diagonalisable si et seulement si \\ la somme des multiplicités géométriques de ses valeurs propres \\ est égale à \(n\).
\end{center}
Une condition suffisante \emph{mais pas nécessaire} est la suivante:
\begin{center}
	Une matrice \(A\) est diagonalisable \\ si elle possède \(n\) valeurs propres \underline{distinctes}.
\end{center}

\section{Orthogonalité}



















\end{document}