\documentclass{article}

\usepackage[utf8]{inputenc}
\usepackage{amsmath}
\usepackage{amssymb}
\usepackage{mathtools}
\newcommand\eqdef{\; \stackrel{\mathclap{\normalfont\text{déf.}}}{ = } \;}
\numberwithin{equation}{section}

\usepackage{array}
\newcolumntype{L}{>{$}l<{$}} % math-mode version of "l" column type
\newcolumntype{C}{>{$}c<{$}} % math-mode version of "c" column type
%the above column types were found on https://tex.stackexchange.com/a/112585

\title{\vspace{-2cm}Analyse 1 - Anna Lachowska\\Résumé}
\author{Benjamin Bovey}
\date{November 2018}

\addtolength{\oddsidemargin}{-.875in}
	\addtolength{\evensidemargin}{-.875in}
	\addtolength{\textwidth}{1.75in}

	\addtolength{\topmargin}{-.875in}
	\addtolength{\textheight}{1.75in}

\begin{document}

\maketitle

\section{Dérivées}
\subsection{Définition}
Les deux définitions suivantes de la dérivée de \(f(x)\) en \(x=x_0\) sont équivalentes:
\begin{align*}
	&f'(x_0) = \lim_{x \to x_0} \dfrac{f(x) - f(x_0)}{x - x_0}
	& &f'(x_0) = \lim_{h \to 0} \dfrac{f(x_0 + h) - f(x_0)}{h}
\end{align*}

\subsection{Dérivées utiles}
\begin{align*}
	&(\sin(x))' = \cos(x)		& &(\cos(x))' = -\sin(x)			& &(\tan(x))' = \frac{1}{\cos^2(x)} \\
	&(x^2)' = 2x 				& &(x^3)' = 3x^2				& &(x^n)' = n \cdot x^{n-1} \\
\end{align*}

\subsection{Opérations sur les dérivées}
Soient les applications \(f: \mathbb{R} \to \mathbb{R}\) et \(g: \mathbb{R} \to \mathbb{R}\) et les réels \(\alpha, \beta \in \mathbb{R}\):
\begin{align*}
	&(\alpha f + \beta g)' = \alpha f' + \beta g' 
	& &(f \cdot g)' = f' \cdot g + f \cdot g' 
	& &\left (\frac{f}{g}\right ) = \frac{f' \cdot g + f \cdot g'}{g^2} \\ 
\end{align*}

\end{document}
