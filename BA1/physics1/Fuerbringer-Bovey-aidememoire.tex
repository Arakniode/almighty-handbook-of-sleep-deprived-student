\documentclass{article}

\title{Aide-mémoire du type qui arrive jamais à passer des concepts théoriques aux équations mathématiques lors de la résolution de problèmes en physique}
\author{Benjamin Bovey - EPFL IC}
\date{Year 2018-2019}

\usepackage[utf8]{inputenc}

\usepackage{amsmath}
%\usepackage{amssymb}
%\usepackage{mathtools}

\begin{document}
\maketitle

\textbf{LORSQU'ON SAIT QU'UNE QUANTITÉ Q EST CONSTANTE} \\
Cela implique bien sûr que \(\boxed{Q(t_1) = Q(t_2)}\). \\

\textbf{SI ON NOUS DEMANDE DE CONFIRMER UNE PROPRIÉTÉ SIMPLE DU SYSTÈME AU DÉBUT DE L'EXO} \\
Cela implique probablement que l'on doit \underline{\emph{utiliser}} cette propriété. \\
Ex: "L'énergie mécanique du système est-elle conservée?" 

\textbf{QUAND IL Y A DES EXPRESSIONS CHELOUS} \\
Regarder si on peut la remplacer par autre chose qu'on a calculé avant. Par exemple (série 11 ex 1)
\begin{align*}
	\omega = \Omega \dfrac{\sqrt{d^2 + R^2}}{R} \\
\end{align*}

\end{document}